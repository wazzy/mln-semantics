\section{Transforming natural language text to logical form}
\index{Boxer}
\index{Discourse Representation Theory}

In transforming natural language text to logical form, we build on the software
package Boxer \citep{bos:coling2004}. Boxer
is an extension to the C\&C parser \citep{clark:acl04} that transforms a parsed
discourse of one or more sentences into a semantic representation.  Boxer
outputs the meaning of each discourse as a Discourse Representation Structure
(DRS) that closely resembles the structures described by \citet{kamp:book93}.

We chose to use Boxer for two main reasons.  First, Boxer is a wide-coverage
system that can deal with arbitrary text.
% that is able to return a reasonable logical representation of most English
% sentences.  Since our goal is to work with actual texts, it is critical that
% we have a wide-coverage semantic parser.  If we did not, then we would be
% unable to deal with any but the simplest texts.
Second, the DRSs that Boxer produces are close to the standard first-order
logical forms that are required for use by the MLN software package Alchemy.
Our system interprets discourses with Boxer, augments the resulting logical
forms by adding inference rules, and outputs a format that the MLN
software Alchemy can read.
